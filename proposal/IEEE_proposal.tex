\documentclass[conference]{IEEEtran}
\IEEEoverridecommandlockouts
% The preceding line is only needed to identify funding in the first footnote. If that is unneeded, please comment it out.
\usepackage{cite}
\usepackage{amsmath,amssymb,amsfonts}
\usepackage{algorithmic}
\usepackage{graphicx}
\usepackage{textcomp}
\usepackage{xcolor}
\def\BibTeX{{\rm B\kern-.05em{\sc i\kern-.025em b}\kern-.08em
    T\kern-.1667em\lower.7ex\hbox{E}\kern-.125emX}}
    
\begin{document}

\nocite{*}

\title{Project Proposal}

\author{\IEEEauthorblockN{Franziska Schwaiger}
    \IEEEauthorblockA{\textit{Matriculation number: 03658670}}
    \and
    \IEEEauthorblockN{Thomas Brunner}
    \IEEEauthorblockA{\textit{Matriculation number: 03675118}}
}

\maketitle

\section*{Objective}

In this project, we will investigate and implement Invertible Neural Networks \cite{Ardizzone2018}.
Our goal is to evaluate their viability and usefulness in problems related to robotics.
More specifically, we want to assess how well INNs can be used to solve inverse kinematics problems
both in planar robotic arms and in complex three-dimensional robotic arms with multiple degrees of freedom.

Real-world robotics applications often involve high-dimensional spaces.
However current INNs architectures become less effective when scaled up to higher dimesionalities
due to limitations in their choice of a discriminator \cite{Ardizzone2018}.

Moreover, current implementations of INNs in the field of robotics have focused on theoretical experiments
and simple simulations in two-dimensional space \cite{Ardizzone2018}.

Due to these limitations, we will investigate how INNs would perform in more complex robotics problems.
Moreover, we will attempt to propose improvements to these models and look for new techniques to allow
INNs to operate on higher-dimensional spaces.

The relevance of our work would be that we would determine the effectiveness of INNs
and similar models in solving inverse kinematics problems in robotics.

If these approaches prove to be successful, they could be used in real-world robotics problems
and would possibly enable the expansion of the use of more complex robotic systems.

\section*{Related Work}

Neural networks have already been used to attempt to solve inverse kinematics problems.
Examples are conditional Variational Autoencoders (cVAE) \cite{Sohn2015}
and conditional Generative Adversarial Networks (cGAN) \cite{Mehdi2018}.

A comparison between INNs and alternatives has been done before \cite{Kruse2019}.
However, existing benchmarks use simple problems with planar robots as a metric,
which do not necesarily translate to real-world robotic applications.

In this project we will expand these comparisons and evaluate the performance of INNs and competing models on
more complex robotics problems.

\section*{Technical Outline}

We address the problem in handling the non-uniqueness of the solution regarding the joint parameters in inverse kinematics by studying generative neural networks. Specifically, we try to estimate the full posterior of the robot joint space with Invertible Neural Networks (INN) \cite{Ardizzone2018}. By introducing additional latent variables in the forward process which follow a certain distribution, INNs also capture the data which is lost during the inverse process resulting in a bijective mapping from TCP coordinates to joint parameters.

In order to analyze the performance of INNs, we simulate the forward kinematics of a planar 3DoF manipulator and generate a dataset which is used for training and testing the performance on estimating the full posterior of the three joint angles.

As a baseline, we also implement a conditional Variational Autoencoder (cVAE) \cite{Sohn2015} which is an extension of VAE where both the input parameters and the latent variables are conditioned on the TCP coordinates.

As INNs use MMD to compare two probability distributions, the authors of \cite{Ardizzone2018} also mention the problem of computing the distance between high-dimensional probability distribtutions, as MMD becomes less effective in this case.

Our contribution is two-fold: First, we want to illustrate this claim by systematically increasing the number of DoF of the planar robot arm and compare the performance of cVAEs and INNs accordingly. Second, we try alternative methods to MMD which also compute the distance between two probability distributions only given samples from these distributions.

\bibliographystyle{plain}
\bibliography{M335}

\end{document}